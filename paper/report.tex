\PassOptionsToPackage{unicode=true}{hyperref} % options for packages loaded elsewhere
\PassOptionsToPackage{hyphens}{url}
%
\documentclass[]{article}
\usepackage{lmodern}
\usepackage{amssymb,amsmath}
\usepackage{ifxetex,ifluatex}
\usepackage{fixltx2e} % provides \textsubscript
\ifnum 0\ifxetex 1\fi\ifluatex 1\fi=0 % if pdftex
  \usepackage[T1]{fontenc}
  \usepackage[utf8]{inputenc}
  \usepackage{textcomp} % provides euro and other symbols
\else % if luatex or xelatex
  \usepackage{unicode-math}
  \defaultfontfeatures{Ligatures=TeX,Scale=MatchLowercase}
\fi
% use upquote if available, for straight quotes in verbatim environments
\IfFileExists{upquote.sty}{\usepackage{upquote}}{}
% use microtype if available
\IfFileExists{microtype.sty}{%
\usepackage[]{microtype}
\UseMicrotypeSet[protrusion]{basicmath} % disable protrusion for tt fonts
}{}
\IfFileExists{parskip.sty}{%
\usepackage{parskip}
}{% else
\setlength{\parindent}{0pt}
\setlength{\parskip}{6pt plus 2pt minus 1pt}
}
\usepackage{hyperref}
\hypersetup{
            pdftitle={Development of an automatic tool for periodic surveillance of actuarial and demographic indicators},
            pdfauthor={Daniel Alonso, María Luz Durbán Reguera, Bernardo D'Auria},
            pdfborder={0 0 0},
            breaklinks=true}
\urlstyle{same}  % don't use monospace font for urls
\usepackage[margin=1in]{geometry}
\usepackage{graphicx,grffile}
\makeatletter
\def\maxwidth{\ifdim\Gin@nat@width>\linewidth\linewidth\else\Gin@nat@width\fi}
\def\maxheight{\ifdim\Gin@nat@height>\textheight\textheight\else\Gin@nat@height\fi}
\makeatother
% Scale images if necessary, so that they will not overflow the page
% margins by default, and it is still possible to overwrite the defaults
% using explicit options in \includegraphics[width, height, ...]{}
\setkeys{Gin}{width=\maxwidth,height=\maxheight,keepaspectratio}
\setlength{\emergencystretch}{3em}  % prevent overfull lines
\providecommand{\tightlist}{%
  \setlength{\itemsep}{0pt}\setlength{\parskip}{0pt}}
\setcounter{secnumdepth}{0}
% Redefines (sub)paragraphs to behave more like sections
\ifx\paragraph\undefined\else
\let\oldparagraph\paragraph
\renewcommand{\paragraph}[1]{\oldparagraph{#1}\mbox{}}
\fi
\ifx\subparagraph\undefined\else
\let\oldsubparagraph\subparagraph
\renewcommand{\subparagraph}[1]{\oldsubparagraph{#1}\mbox{}}
\fi

% set default figure placement to htbp
\makeatletter
\def\fps@figure{htbp}
\makeatother

\usepackage{setspace}\doublespacing

\title{Development of an automatic tool for periodic surveillance of actuarial
and demographic indicators}
\author{Daniel Alonso, María Luz Durbán Reguera, Bernardo D'Auria}
\date{July-August 2021}

\begin{document}
\maketitle

{
\setcounter{tocdepth}{2}
\tableofcontents
}
\newpage

\hypertarget{abstract}{%
\section{Abstract}\label{abstract}}

As a result of the COVID-19 pandemic and its large impact across Spain,
the monitoring of demographic measures as a direct result of deaths
related to such pandemic and future similarly deadly events has become
increasingly important. It is intended with this project to develop a
tool in order to easily monitor a selection of demographic measures
relating to collective deaths of individuals as a result of relevant
worldwide events like the one mentioned previously. The tool consists of
a shiny dashboard (developed in R) where such measures are displayed in
different visualizations across time.

\hypertarget{introduction}{%
\section{Introduction}\label{introduction}}

The COVID-19 pandemic has led to a widespread and noticeable temporary
increase in mortality and reduction in life expectancy throughout Spain.
This arises a need to monitor these demographic measures more closely
and in real time.

This project consists of a shiny dashboard with several features:

\begin{itemize}
\item
  Visualizing several mortality metrics:

  \begin{itemize}
  \tightlist
  \item
    Excess mortality
  \item
    Cumulative mortality rate
  \item
    Cumulative relative mortality rate
  \item
    Mortality improvement factor
  \end{itemize}
\item
  Visualizing life expectancy and constructing life tables
\item
  Visualizing a map of Spain with the previous metrics per autonomous
  community (CCAA)
\end{itemize}

All metrics are calculated weekly with data stretching back as far back
as 2010.

\hypertarget{objectives}{%
\section{Objectives}\label{objectives}}

\begin{itemize}
\item
  Provide a simple-to-use, web-based, OS-agnostic tool to compute and
  visualize common mortality and life expectancy metrics in time series
  plots/maps
\item
  Provide the user the ability to customize the plot parameters
  significantly
\item
  Provide the user the ability to download the plots and the data (with
  or without filtering)
\item
  Allow the user to update and push the data to the corresponding github
  repository hosting the data from within the application
\item
  Have data updated in real-time from the official Spanish sources and
  Eurostat (also provided by INE)
\end{itemize}

\hypertarget{project-repository-tree-structure}{%
\section{Project repository tree
structure}\label{project-repository-tree-structure}}

First of all, the project consists of two repositories:

\begin{itemize}
\item
  🗄️ \href{https://github.com/dreth/tfm_uc3m}{The main project
  repository (\emph{tfm\_uc3m})}
\item
  🗄️ \href{https://github.com/dreth/tfm_c3m_data}{The data repository
  (\emph{tfm\_uc3m\_data})}
\end{itemize}

\end{document}
